\documentclass[../main]{subfiles}

\begin{document}
\chapter{C++14}
\section{Basics}
\subsection{\texttt{auto} as a result of function}
    From now the compiler can deduce a returning type of the expression
\begin{Code}
    #include <iostream>
    
    template <typename T1, typename T2>
    auto multiply(T1 t1, T2 t2)
    {
        return t1 * t2;
    }
    
    int main()
    {
        std::cout << multiply(1, 2.0) << std::endl;
    
        return 0;
    }
\end{Code}
\noindent
If there is more than one \texttt{return} statement, the compiler will check their consistency - if they are not the same, compilation will fail
(even if there is a conversion from one type to another)
\begin{Code}
    #include <iostream>
    
    template <typename T1, typename T2>
    auto max(T1 t1, T2 t2)
    {
        if (t1 < t2)
        {
            return t2;
        }
        return t1;
    }
    
    int main()
    {
        /* error: inconsistent deduction for auto return type */
        // std::cout << max(1, 2.0) << std::endl;
    
        return 0;
    }
\end{Code}

    In \ref{Perfect returning} subchapter we will discuss the way of using \texttt{decltype(auto)}, also available since C++14. The only thing
    we have to keep in mind is that \texttt{decltype} gives the declared type of the expression that is passed to it. The \texttt{auto}
    keyword does the same thing as template type deduction so, for example, if you have a function that returns a reference, \texttt{auto} will
    still be a value (you need \texttt{auto\&} to get a reference), but \texttt{decltype} will be exactly the type of the return value.

\subsection{The variable template}
    C++ introduces, in addition to class and function templates, the template of variables. Beyond the ability to avoid
casting (by creating the variable of perfectly matching type), we can employ them to type traits. Since now, we don't need
to create special structures with \texttt{value} member - we just can have a single variable.
\begin{Code}
    #include <iostream>
    #include <string>
    #include <type_traits>
    
    /* For any given type be false */
    template <typename T>
    /* Must be constexpr ! */
    constexpr bool is_string_v = false;
    
    /* For string be true */
    template <>
    constexpr bool is_string_v<std::string> = true;
    
    template <typename T,
              std::enable_if_t<is_string_v<T>, bool> = true>
    void info(const T&)
    {
        std::cout << "std::string" << std::endl;
    }
    
    template <typename T,
              std::enable_if_t<!is_string_v<T>, bool> = true>
    void info(const T&)
    {
        std::cout << "not std::string" << std::endl;
    }
    
    int main()
    {
        /* Prints "not std::string" */
        info(int {});

        /* Prints "std::string" */
        info(std::string {});
    
        return 0;
    }
\end{Code}

\section{Functions}
\subsection{A \texttt{constexpr} function}
    Starting from C++14, many of the constraints of \texttt{constexpr} functions are relaxed. Since now we can
\begin{itemize}
    \item use loops,
    \item use \texttt{switch} and \texttt{if} statements,
    \item use define own \texttt{constexpr} variables inside the body of \texttt{constexpr} function.
\end{itemize}

    Let's get back to the example \ref{Binary numbers} and rewrite it with our new opportunities
\begin{Code}
    constexpr std::size_t to_int(char c)
    {
        return c - 48;
    }
    
    constexpr std::size_t bin_power(std::size_t N)
    {
        if (N == 0)
        {
            return 1;
        }
        
        std::size_t value { 1 };
        for (int i = 0; i < N; ++i)
        {
            value *= 2;
        }
        return value;
    }
    
    constexpr std::size_t to_dec_base(const char* arr, std::size_t size)
    {
        std::size_t acc { 0 };
        for (std::size_t i = 0; i < size; i++)
        {
            acc += bin_power(size - 1 - i) * to_int(arr[i]);
        }
        return acc;
    }
    
    constexpr std::size_t operator"" _bin(const char* bits)
    {
        if (bits[0] == '\0')
        {
            return 0;
        }
        
        std::size_t size { 0 };
        while (bits[size] != '\0')
        {
            size++;
        }
        return to_dec_base(bits, size);
    }

    /* IDE prints "9" */
    constexpr std::size_t dec = 1001_bin;
\end{Code}

\subsection {Lambdas}
    The C++14 standard extends lambda capabilities to have \textbf{universal type of paramenter}, for instance
\begin{Code}
    /* Lambda can have universal parameter type */
    [](auto u, auto v) { return u + v; };
\end{Code}
\noindent
The mechanism behind the universal lambda is that the compiler converts it into the functor with the template \texttt{operator()} function, thus
in our case, it could be
\begin{Code}
    struct Lambda
    {
        template <typename T1, typename T2>
        auto operator(T1 t1, T2 t2) { return t1 + t2; }
    };
\end{Code}

    Apart from universal parameters, C++14 gives us one more convenient tool which is initialization in the capture list
\begin{Code}
    #include <iostream>
    
    int main()
    {
        int value = 1;
        
        /* Named variable in the capture list */
        auto lambda = [shift = value * 3](auto u, auto v)
        {
            return u + v + shift;
        };
        
        /* Prints "6" */
        std::cout << lambda(1, 2) << std::endl;
    
        return 0;
    }
\end{Code}
\noindent
That is done by adding a member to the functor
\begin{Code}
    #include <iostream>
    
    template <typename Shift>
    struct Lambda
    {
        explicit Lambda(Shift shift):
            m_shift(shift)
        {
        }
        
        template <typename T1, typename T2>
        auto operator()(T1 t1, T2 t2)
        {
            return t1 + t2 + m_shift;
        }
        
        Shift m_shift;
    };
    
    int main()
    {
        int value = 1;
        
        /* Prints "6" */
        std::cout << Lambda<decltype(value)> { value * 3 } (1, 2)
                  << std::endl;
        return 0;
    }
\end{Code}

    What is more, the lambda expression supports move semantics
\begin{Code}
    #include <iostream>
    
    struct Data
    {
        Data(): m_value(0)
        {
            std::cout << "Data" << std::endl;
        }
        
        Data(const Data& data): m_value(data.m_value)
        {
            std::cout << "const Data&" << std::endl;
        }
        
        Data(Data&& data): m_value(std::move(data.m_value))
        {
            data.m_value = 0;
            std::cout << "Data&&" << std::endl;
        }
        
        ~Data()
        {
            std::cout << "~Data" << std::endl;
        }
        
        void set(int value)
        {
            m_value = value;
        }
        
        int m_value;
    };
    
    int main()
    {
        /* Prints "Data" */
        Data data;
        
        /* Data is moved even if the lambda is not called! */
        /* Prints "Data&&" */
        auto lambda = [moved_data = std::move(data)] () mutable
        {
            moved_data.set(1);
            return moved_data;
        };
        
        /* Prints "const Data&" */
        auto&& result = lambda();
        
        /* Prints "0" */
        std::cout << data.m_value << std::endl;
        
        /* Prints "1" */
        std::cout << result.m_value << std::endl;
        
        return 0;
    }
\end{Code}

\section{Miscellaneous}
    C++14 provided a bit more useful elements
\begin{enumerate}
    \item New attributes like \texttt{[[noreturn]]}, \texttt{[[deprecated]]} etc.
    \item A separators in numbers \texttt{int value = 500`000`000}.
    \item \texttt{new} and \texttt{delete} operators overloading.
\end{enumerate}
\end{document}
