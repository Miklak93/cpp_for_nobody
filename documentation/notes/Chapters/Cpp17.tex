\documentclass[../main]{subfiles}

\begin{document}
\chapter{C++17}
\section{Basics}
\subsection{\texttt{if} and \texttt{switch} with initialization}
    Since C++17 both \texttt{if} and \texttt{switch} statements can be equipped with \textbf{initialization}.
\begin{Code}
    #include <iostream>
    #include <random>
    #include <string>
    #include <utility>
    
    const int EXPECTED = 2;
    
    int get_random()
    {
        std::random_device device;
        std::mt19937 engine { device() };
        std::uniform_int_distribution<std::mt19937::result_type> dstr(0, 4);
        
        return dstr(engine);
    }
    
    inline bool more_than(int value, int expected)
    {
        return value > expected;
    }
    
    int main()
    {
        /* Initialization in if statement */
        if (int value = get_random(); more_than(value, EXPECTED)) 
        {
            std::cout << value << " > " << EXPECTED << std::endl;
        }
        else
        {
            std::cout << value << " < " << EXPECTED << std::endl;
        }

        /* Initialization in switch statement */
        switch (int value = get_random(); more_than(value, EXPECTED))
        {
            case true:
                std::cout << value << " > " << EXPECTED << std::endl;
                break;
            default:
                std::cout << value << " < " << EXPECTED << std::endl;
        }
    
        return 0;
    }
\end{Code}

\subsection{\texttt{constexpr if}}
    The compile-time \texttt{if} opens the door to use conditional statements during compilation. The big difference between standard \texttt{if} is that we usually need to use
\texttt{else} complementary part explicitly.
\begin{Code}
    #include <string>
    
    template <int value>
    auto function()
    {
        if constexpr (value > 0)
        {
            return value; 
        }
        return std::string { "0" };
    }
    
    int main()
    {
        function<0>();
        
        /* Won't compile */
        // function<1>();
    
        return 0;
    }
\end{Code}
\noindent
Observe that the condition \texttt{value < 0} discards \texttt{if constexpr} so \texttt{std::string} is the only one deduced type. The situation is drastically different when we
use \texttt{value > 0} - in that case, \texttt{if constexpr} is fulfilled, and then two different types i.e. \texttt{int} and \texttt{std::string} are deduced as
return value. This is the reason for the failure. To correct it, we need to add the \texttt{else} clause

\begin{Code}
    #include <string>
    
    template <int value>
    auto function()
    {
        if constexpr (value > 0)
        {
            return value; 
        }
        else
        {
            return std::string { "0" };
        }
    }
    
    int main()
    {
        std::string str = function<0>();
        int i = function<1>();
        
        return 0;
    }
\end{Code}
\noindent
When \texttt{value > 0} the condition \texttt{if constexpr} is fulfilled but \texttt{else} discarded, thus there is no conflict for the return value.\newline

    Besides making \texttt{type\_traits} much more convenient to write, it allows to construct functions with different return type
\begin{Code}
template <int value>
auto function()
{
    if constexpr (value > 0)
    {
        return value;
    }
}
\end{Code}
\noindent
This function returns either \texttt{int} or \texttt{void} depending on whether \texttt{value} is positive or not. As a for regular \texttt{if},
its compile-time counterpart supports the initialization statement.\newline

    At the end, it is, of course, possible to use \texttt{if else} clause
\begin{Code}
    #include <iostream>
    #include <type_traits>
    
    template <typename T>
    void check_type()
    {
        if constexpr (std::is_same_v<T, int>)
        {
            std::cout << "int" << std::endl;
        }
        else if constexpr (std::is_same_v<T, float>)
        {
            std::cout << "float" << std::endl;
        }
        else if constexpr (std::is_same_v<T, double>)
        {
            std::cout << "double" << std::endl;
        }
        /* Required ! */
        else
        {
            std::cout << "Unknown type" << std::endl;
        }
    }
    
    int main()
    {
        /* Prints "int" */
        check_type<int>();
        
        /* Prints "float" */
        check_type<float>();
        
        /* Prints "double" */
        check_type<double>();
        
        /* Prints "Unknown type" */
        check_type<char>();
    
        return 0;
    }
\end{Code}

\section{Functions}
    From now lambdas achieve a few more possibilities
\begin{itemize}
    \item they are \texttt{constexpr} by default and if possible we can use them as usual \texttt{consexpr} functions.
    \begin{Code}
        #include <iostream>
        #include <type_traits>
        
        auto is_negative = [](auto value) { return value < 0; };
        
        template <int value,
                  std::enable_if_t<is_negative(value), bool> = true>
        int absolute_value()
        {
            return -value;
        }
        
        template <int value,
                  std::enable_if_t<!is_negative(value), bool> = true>
        int absolute_value()
        {
            return value;
        }
        
        int main()
        {
            /* Prints "1" */
            std::cout << absolute_value<-1>() << std::endl;

            /* Prints "0" */
            std::cout << absolute_value<0>() << std::endl;
            
            /* Prints "1" */
            std::cout << absolute_value<1>() << std::endl;
        
            return 0;
        }
        \end{Code}
    \item In C++11 and C++14 passing [\&], [=] or [this] always allows to modify class, because the copy of the \texttt{this} pointer. C++17 fills this gap
    by introducing \texttt{[*this]} expression
    \begin{Code}
        #include <iostream>

        struct Data
        {
            int value { 0 };
            
            void modify()
            {
                /* Instant call of lambda */
                [copy=*this] () mutable { copy.value = 1; } ();
            }
        };
        
        int main()
        {
            Data data;
            data.modify();
            
            /* Remains untouched so prints "0" */
            std::cout << data.value << std::endl;
        
            return 0;
        }
    \end{Code}
    \noindent
    Remember about it when you're working with multithreading.
\end{itemize}

\section{Classes}
\subsection{Inline variables}
    Before C++17 there was no way to initialize \textbf{non-\texttt{const}}
    \footnote{For \texttt{const} case we can initialize member inside.}
static members inside a class if this class is in the header file

    \begin{Code}
        struct Struct
        {
            static int counter {};
        };

        /* Compilation gives error
         * "error: ISO C++ forbids in-class initialization of
         * non-const static member ‘Struct::counter’"
         */
        int main()
        {
            return 0;
        }
    \end{Code}

    We could move the definition outside of the class and that will
work
    \subsubsection{\textit{header.h}}
    \begin{Code}
        #pragma once
        
        struct Struct
        {
            static int counter;
        };

        int Struct::counter = 0;
    \end{Code}
    
    \subsubsection{\textit{main.cpp}}
    \begin{Code}
        #include "header.h"
        
        int main()
        {
            return 0;
        }
    \end{Code}
but what if we include the same header in another \textit{.cpp} file ?
    \subsubsection{\textit{includer.cpp}}
    \begin{Code}
        #include "header.h"
    \end{Code}
In such a case the compilation will end with error
\textit{multiple definition of `Struct::counter'}.\newline

    Since C++17 we have a solution in the form of
\textbf{\texttt{inline} variable}
\begin{Code}
    #include <iostream>
    
    struct Struct
    {
        Struct()
        {
            std::cout << "Struct " << ++counter << " created\n";
        }
        
        ~Struct()
        {
            std::cout << "Struct " << counter-- << " destroyed\n";
        }

        /* static inline variable */
        static inline int counter { 0 };
    };
    
    int main()
    {
        /* Prints "Struct 1 created" */
        Struct s1;
        {
            /* Prints "Struct 2 created" */
            Struct s2;
            {
                /* Prints "Struct 3 created" */
                Struct s3;
            }
            /* Prints "Struct 3 destroyed" */
        }
        /* Prints "Struct 2 destroyed" */
        
        return 0;
        /* Prints "Struct 1 destroyed" */
    }
\end{Code}
That could be handy with \texttt{thread\_local} keyword - threads won't share the same object anymore. Moreover, \texttt{constexpr}
implies \texttt{inline}, thus
\begin{Code}
    static inline constexpr int value { 0 };
\end{Code}
means the same as
\begin{Code}
    static constexpr int value { 0 };
\end{Code}

\subsection{Aggregate}
    Since C++17, aggregates can have base classes, so that for such structures being derived from other
classes or structures list initialization is allowed. From now the definition of aggregate changed and now the \textit{aggregate} means
\begin{itemize}
    \item either an array,
    \item or a class type (class, struct, or union) with:
    \begin{itemize}
        \item no user-declared or explicit constructor
        \item no constructor inherited by a \texttt{using} declaration
        \item no private or protected non-static data members
        \item no virtual functions
        \item no virtual, private, or protected base classes
    \end{itemize}
\end{itemize}

\begin{Code}
    #include <string>
    
    struct AggregateBase
    {
        int i;
        float f;
    };
    
    struct AggregateDerived : AggregateBase
    {
        std::string string;
    };
    
    int main()
    {
        /* Base class can have separate parenthesis */
        AggregateDerived derived0 { { 0, 0.0 }, "str0" };
        
        /* But we don't need to use them */
        AggregateDerived derived1 { 1, 1.1, "str1" };
        
        /* We can skip initial values - missing ones
         * will be zero-initialized
         */
        AggregateDerived derived3;
        AggregateDerived derived4 { 4, 4.4 };
        AggregateDerived derived5 { { 5 }, "str5" };
        
        return 0;
    }
\end{Code}

    Aggregates extend over derived classes from non-aggregates. This
isn't a surprise as a derived class ''holds'' the base class object
inside 
\begin{Code}
    #include <string>
    
    struct AggregateString : std::string
    {
        int i;
    };
    
    int main()
    {
        AggregateString aggregate0 { { "str0" }, 0 };
        AggregateString aggregate1 { {  }, 1 };
        
        return 0;
    }
\end{Code}
\noindent
or even multiple classes
\begin{Code}
    #include <string>
    #include <vector>
    
    struct AggregateMultipleBase : std::string, std::vector<int>
    {
        int i;
    };
    
    int main()
    {
        AggregateMultipleBase derived0 { { "str0" }, { 1, 2, 3 }, 0 };
        AggregateMultipleBase derived1 { "str1", {},  1 };
        
        return 0;
    }
    
\end{Code}

\subsection{Structured bindings}
    Structured bindings can be used for structures with public data members, raw C-style arrays, and \textit{tuple-like objects:}
\begin{itemize}
    \item If in structures and classes, all non-static data members are public, you can bind each non-static
    data member to exactly one name.
    \item For raw arrays, you can bind a name to each element.
    \item For any type you can use a tuple-like API to bind names to whatever the API defines as “elements.”
    The API roughly consists of the following elements for a type type:
    \begin{itemize}
        \item \texttt{std::tuple\_size<type>::value} has to return the number of elements.
        \item \texttt{std::tuple\_element<idx,type>::type} has to return the type of the idx-th element.
        \item A global or member \texttt{get<idx>()} has to yield the value if the idx-th element.
    \end{itemize}
    The standard library types \texttt{std::pair}, \texttt{std::tuple}, and \texttt{std::array} already provide this API.
\end{itemize}

    To understand structured bindings, be aware that there is a new anonymous
variable involved. The new names introduced as structure bindings refer to members or elements of this
anonymous object. The exact behavior of an initialization
\begin{Code}
    struct Data
    {
        type1 i;
        type2 s;
    } data;
    
    auto [u,v] = data;
\end{Code}
is as we’d initialize a new entity \texttt{entity} with \texttt{data} and let the structured bindings \texttt{u} and \texttt{v} become
alias names for the members of this new object, similar to defining
\begin{Code}
    auto entity = data;
    aliasname u = entity.i;
    aliasname v = entity.s;
\end{Code}
\noindent
Note that \texttt{u} and \texttt{v} are not references to \texttt{entity.i} and \texttt{entity.s}, respectively. They are just other names for the
members. Thus, \texttt{decltype(u)} is the type of the member \texttt{i} and \texttt{decltype(v)} is the type of the member \texttt{s}.

\subsubsection{Qualifiers}
    We can use qualifiers, such as \texttt{const} or \texttt{volatile} and references. Again, these qualifiers apply to the anonymous entity as a whole
\begin{Code}
    const auto& [u,v] = data;
\end{Code}
\noindent
Qualifiers don’t necessarily apply to the structured bindings. Both
\texttt{u} and \texttt{v} are declared as references and this only
specifies that the anonymous entity is a reference -
\texttt{u} and \texttt{v} have the type of the members of
\texttt{data}.\newline

    Structured bindings \textbf{do not \textit{decay}}
\footnote{The \textit{decay} is the type same as when arguments are 
passed by value, which means that raw arrays convert to pointer and
top-level qualifiers, such as \texttt{const} and references, are ignored.}
although \texttt{auto} is used. Look at the following example
\begin{Code}
    struct Struct
    {
        const char x[6];
        const char y[3];
    };
    
    Struct s{};
    auto [u, v] = s;
\end{Code}
the type of \texttt{u} and \texttt{v} is \texttt{const char[6]} and 
\texttt{const char[3]} respectively since
\textbf{\texttt{auto} doesn't decay in  this case}
\footnote{Compare it with \ref{Auto and pointers} section}.
\begin{Code}
    #include <iostream>
    #include <type_traits>
    
    struct Struct
    {
        const char x[6];
        const char y[3];
    };
    
    int main()
    {
        Struct s{};
        auto [u, v] = s;
        
        /* Prints "1" */
        std::cout << std::is_same_v<decltype(u), const char[6]> << std::endl;
        
        /* Prints "1" */
        std::cout << std::is_same_v<decltype(v), const char[3]> << std::endl;
        return 0;
    }    
\end{Code}

\subsubsection{Structured binding with classes}
    Note that there is only limited usage of inheritance possible - all non-static data members must be members of the same class definition. Therefore, they have to be direct members of the type or
the same unambiguous public base class
\begin{Code}
    struct Base
    {
        int a {1};
        int b {2};
    };

    struct Derived : Base {};

    int main()
    {
        /* Copy of values - everything works */
        auto [u, v] = Derived{};
        return 0;
    }
\end{Code}
\noindent
The code above compiles without any problem, but
\begin{Code}
    struct Base
    {
        int a {1};
        int b {2};
    };

    struct Derived : Base
    {
        int c = {3};
    };

    int main()
    {
        /* Won't compile */
        // auto [u, v, z] = Derived {};
    }
\end{Code}
do not compile - the error says \textit {cannot decompose class type ‘Derived’: both it and its base class ‘Base’ have non-static data members}.\newline

    The C++ language provides a way how to manually implement structured bindings for user-defined types.
\begin{Code}
    #include <iostream>
    #include <string>
    #include <utility>
    
    class Class
    {
    public:
        Class(std::string str1, std::string str2, long val):
            m_str1 ( std::move(str1) ),
            m_str2 ( std::move(str2) ),
            m_val ( val )
    {
    }
    
    const std::string& get_str1() const
    {
        return m_str1;
    }
    
    std::string& get_str1()
    {
        return m_str1;
    }
    
    const std::string& get_str2() const
    {
        return m_str2;
    }
    
    std::string& get_str2()
    {
        return m_str2;
    }
    
    const long& get_val() const
    {
        return m_val;
    }
    
    long& get_val()
    {
        return m_val;
    }
        
    private:
        std::string m_str1;
        std::string m_str2;
        long m_val;
    };
    
    /* Most important part - specialization for user-defined class */
    template <>
    struct std::tuple_size<Class>
    {
        static constexpr std::size_t value = 3;
    };
    
    template <std::size_t Idx>
    struct std::tuple_element<Idx, Class>
    {
        using type = std::string;
    };
    
    template <>
    struct std::tuple_element<2, Class>
    {
        using type  = long;
    };
    
    /* The function name must be get!
     * decltype(auto) is crucial. Firstly, auto is needed
     * since we return different types from this function.
     * Secondly, decltype(auto) do not decay - we return
     * references here.
     */
    template <std::size_t N>
    decltype(auto) get(Class& c)
    {
        static_assert(N < 3, "Class has only three elements!");
        if constexpr (N == 2)
        {
            return c.get_val();
        }
        else if constexpr (N == 1)
        {
            return c.get_str2();
        }
        else
        {
            return c.get_str1();
        }
    }
    
    template <std::size_t N>
    decltype(auto) get(const Class& c)
    {
        static_assert(N < 3, "Class has only three elements!");
        if constexpr (N == 2)
        {
            return c.get_val();
        }
        else if constexpr (N == 1)
        {
            return c.get_str2();
        }
        else
        {
            return c.get_str1();
        }
    }
    
    template <std::size_t N>
    decltype(auto) get(Class&& c)
    {
        static_assert(N < 3, "Class has only three elements!");
        if constexpr (N == 2)
        {
            return std::move(c.get_val());
        }
        else if constexpr (N == 1)
        {
            return std::move(c.get_str2());
        }
        else
        {
            return std::move(c.get_str1());
        }
    }
    
    int main()
    {
        Class c { "str1", "str2", 0 };
        
        const auto& [cstr1, cstr2, cval] = c;
        
        /* Prints "str1" */
        std::cout << cstr1 << std::endl;
        
        /* Prints "str2" */
        std::cout << cstr2 << std::endl;
        
        /* Prints "0" */
        std::cout << cval << std::endl;
        
        auto& [str1, str2, val] = c;
        str1 = "mod str1";
        str2 = "mod str2";
        val = 1;
        
        auto&& [mstr1, mstr2, mval] = c;
        
        /* Prints "mod str1" */
        std::cout << mstr1 << std::endl;
        
        /* Prints "mod str2" */
        std::cout << mstr2 << std::endl;
        
        /* Prints "mod str1" */
        std::cout << mval << std::endl;
    
        return 0;
    }
\end{Code}

\subsection{Template argument deduction}
    Before C++17 template argument deduction was allowed only for functions. In C++17 this constraint was relaxed and now a compiler
can deduce class template parameters as well.
\begin{Code}
    std::vector vector {1, 2};
\end{Code}
\noindent
    This will not work if we use different types, since
\texttt{std::vector<T>} expects a homogenous type
\begin{Code}
    std::vector vector {1, 2.5, "str"};
\end{Code}
but will work for a variadic-template types
\begin{Code}
    std::tuple { 1, 2.5, "str"};
\end{Code}

    Not all differences between functions and classes are resolved in C++17 - it still doesn't support partial template argument
deduction but there is a question if it should, since in the following example
\begin{Code}
    std::tuple<int> tuple { 0, 1 };
\end{Code}
we don't know whether it was intentional or accidental to pass two arguments but define just one type.\newline

    \textbf{Type deduction prefers copy initialization if some expression could be interpreted that way}
\begin{Code}
    std::vector vector1 {1, 2};
    /* vector2 = std::vector<int> { 1, 2 } so copy */
    std::vector vector2 { vector1 };
\end{Code}
\noindent
When the expression cannot be interpreted as a copy, C++17 uses type deduction
\begin{Code}
    std::vector vector1 {1, 2};
    /* vector2 = std::vector<std::vector<int>> { vector1, vector1 }; */
    std::vector vector2 { vector1, vector1 };
\end{Code}

    There is one corner case. Let's look at the example given below
\begin{Code}
    #include <vector>
    
    template <typename... Args>
    auto make_vector(const Args&... args)
    {
        return std::vector { args... };
    }
    
    int main()
    {
        std::vector vector1 { 1, 2 };
        /* Is it the copy or the vector of vectors? */
        std::vector vector2 = make_vector(vector1);
    
        return 0;
    }
\end{Code}
In such cases, a compiler producer can decide what to do, as the standard does not define the exact rules to follow.\newline

\subsubsection{Deduction guides}
    Due to the automatic deduction in classes, helper functions like \texttt{std::make\_pair}, \texttt{std::make\_unique}, which were designed for
type deduction, seem to be redundant. But they still do something more than type deduction - they \textbf{decay}.

\begin{Code}
    /* [auto] = std::pair<const char*, const char*> */
    auto pair = std::make_pair("str1", "str2");
\end{Code}
\noindent
Let's have a look at the naive implementation of std::pair

\begin{Code}
    template <typename T1, typename T2>
    struct Pair
    {
        Pair(const T1& t1, const T2& t2): m_first(t1), m_second(t2)
        {
        }
        
        T1 m_first;
        T2 m_second;
    };
    
    int main()
    {
        /* Won't compile */
        // Pair p {"str1", "str2"};
    
        return 0;
    }
\end{Code}

    The program above reveals the nature of \texttt{auto} - \textbf{it decays if only it is working on \texttt{values} but not \texttt{references}}. Because
of that, the code presented before is internally working like that
\begin{Code}
    /* Passing parameters phase */
    const char arg1[5] {"str1"};
    const char arg2[5] {"str2"};

    /* Initialization - can't be done that way so we got error */
    char[5] m_first { arg1 }; 
    char[5] m_second { arg2 };
\end{Code}
\noindent
There won't be an error if we pass by the value which decays.\newline

    On the other hand, notice that \texttt{std::pair} is somehow working with references. It is possible because of
\textbf{deduction guides}.

\begin{Code}
template <typename T1, typename T2>
struct Pair
{
    Pair(const T1& t1, const T2& t2): m_first(t1), m_second(t2)
    {
    }
    
    T1 m_first;
    T2 m_second;
};

/* Deduction guide */
template <typename T1, typename T2>
/* No references, so argument will decay */
Pair(T1, T2) -> Pair<T1, T2>;

int main()
{
    Pair p {"str1", "str2"};

    return 0;
}
\end{Code}

    Deduction guides compete with the constructors of a class. Class template argument deduction uses the constructor guide that
has the highest priority according to overload resolution. \textbf{If a constructor and a deduction guide match equally well,
the deduction guide is preferred}. Otherwise, for instance, when no conversion is needed, the constructor is taken.
\begin{Code}
    #include <string>
    
    template <typename T>
    struct Struct
    {
        T value;
    };
    
    Struct(const char*) -> Struct<std::string>;
    
    int main()
    {
        /* It is std::string inside now */
        Struct { "value" }; 
    
        return 0;
    }
\end{Code}

\section{Fold expressions}
    From now on we have two basic ways of using \textbf{fold expressions}. As they are counter-intuitive, they must be memorized
\begin{enumerate}
    \item \textbf{left fold}: \texttt{... \textbf{op} args} which is
    \begin{equation*}
        (\;(\;(\;arg1\; \textbf{op}\; arg2\;)\; \textbf{op}\; arg3\; )\; op\; ...\;)
    \end{equation*}
    \item \textbf{right fold}: \texttt{args \textbf{op} ...} which is
    \begin{equation*}
        (\;arg1\; \textbf{op}\; (\;arg2\; \textbf{op}\; (\;arg3\; \textbf{op}\; (\;...\;)\;)\;)\;)
    \end{equation*}
\end{enumerate}
\noindent
The following program can make it clear
\begin{Code}
    #include <iostream>

    template <typename... Args>
    auto left_sum (const Args&... args)
    {
        return (... + args); // ( ( ( arg1 + arg2 ) + arg3 ) + ... )
    }
    
    template <typename... Args>
    auto right_sum (const Args&... args)
    {
        return (args + ...); // ( arg1 + ( arg2 + ( arg3 + (...) ) )
    }
    
    int main()
    {
        /* Prints "0" */
        std::cout << left_sum(1, 2.0, -3) << std::endl;
        
        /* Prints "0" */
        std::cout << right_sum(1, 2.0, -3) << std::endl;
    
        return 0;
    }
\end{Code}

    Sometimes left or right fold matters. Assume we've invoked \texttt{left\_sum} function like
\begin{Code}
    left_sum("str1", "str2", std::string {"str3"});
\end{Code}
the code won't compile - the expression is unfolded to
\begin{center}
    \texttt{((const char* + const char*) + std::string)}
\end{center}
which doesn't work since \texttt{const char* + const char*} is not defined. On the other hand
\begin{Code}
    right_sum("str1", "str2", std::string {"str3"});
\end{Code}
compiles, because the expression unfolded is
\begin{center}
    \texttt{const char* + (const char* + std::string)}
\end{center}
and this equals
\begin{center}
    \texttt{(const char* + std::string)} = \texttt{std::string}.
\end{center}

We should remember that comma operator \textbf{not always must be used!}.
Let's have a look at this example
\begin{Code}
    template <std::size_t N>
    using thread_pool = std::array<std::thread, N>;
    
    template <typename... Functions>
    auto create_threads(Functions&&... functions)
    {
        return thread_pool<sizeof...(Functions)>
        {
            std::thread{ std::forward<Functions>(functions) }...
        };
    }
\end{Code}
\noindent
If we used
\begin{Code}
    return thread_pool<sizeof...(Functions)>
    {
        /* Comma operator used */
        std::thread{ std::forward<Functions>(functions) }, ...
    }; 
\end{Code}
we would get only \textbf{only last thread}. This is because of the comma
operator behavior - as we already know, the comma operator
\texttt{((expr1, expr2, ...))} evaluates all its operands
in left-to-right order and returns the result of the last operand.
In this case, we're constructing \texttt{std::thread} objects,
but we're not actually returning them or storing them in the array properly.
The result of this expression is just the last \texttt{std::thread} object,
not a sequence of threads.

\subsection{Deal with empty parameters pack}
    Neither \texttt{left\_sum} or \texttt{right\_sum} work with empty parameters pack. That can be fixed with some trick
\begin{Code}
    template <typename... Args>
    auto left_sum (const Args&... args)
    {
        /*  ( ( ( arg1 + arg2 ) + arg3 ) + ... + 0 ) */
        return (0 + ... + args);
    }
    
    template <typename... Args>
    auto right_sum (const Args&... args)
    {
        /* ( 0 + ( arg1 + ( arg2 + ( arg3 + (...) ) ) ) ) */
        return (args + ... 0);
    }
\end{Code}
\noindent
We've added explicit value which will be returned if an empty parameter pack is passed.\newline

    Both compile-time \texttt{if} and fold expressions shall be used with \texttt{type\_traits} to reduce the amount of the code.

\section{Miscellaneous}
    C++17 provided some more elements than discussed above, but they are ,,one-liners'' which can be easily found in the preference. They are here just to point out their existence
\begin{enumerate}
    \item New attributes like \texttt{[[fallthrough]]}, \texttt{[[nodiscard]]} etc.
    \item Defined expression evaluation order.
    \item Nested namespaces.
    \item New classes \texttt{std::optional}, \texttt{std::variant}, \texttt{std::any}, \texttt{std::byte},\newline
    \texttt{std::string\_view}, etc.
    \item New algorithms \texttt{std::size()}, \texttt{std::empty()}, \texttt{std::clamp()}, etc.
    \item \texttt{std::vector}, \texttt{std::list}, \texttt{std::forward\_list} support incomplete types.
\end{enumerate}
\end{document}
